% Glossary

% #1: Id (uima)
% #2: Shortname (UIMA)
% #3: Longname (Unstructured Information Management Applications)
% #4: Description (UIMA are software systems that...)
\newcommand{\newacr}[4]{\newglossaryentry{glos:#1}{name={#3},description={#4}} \newacronym{acr:#1}{#2}{\gls{glos:#1}} }
% TODO: Im Text wird die Abkürzung mehrfach angezeigt.

\newacr{uima}{UIMA}{Unstructured Information Management Applications}{UIMA are software systems that analyze large volumes of unstructured information in order to discover knowledge that is relevant to an end user. An example UIM application might ingest plain text and identify entities, such as persons, places, organizations; or relations, such as works-for or located-at.}

\newcommand{\uima}{\gls{acr:uima}}

\newglossaryentry{glos:dkpro}{%
    name=DKPro,
    description={DKPro is a community of projects focussing on re-usable Natural Language Processing software}
    }

\newcommand{\dkpro}{\gls{glos:dkpro}}
