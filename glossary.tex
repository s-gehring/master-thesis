% Glossary

% #1: Id (uima)
% #2: Shortname (UIMA)
% #3: Longname (Unstructured Information Management Applications)
% #4: Description (UIMA are software systems that...)
\newcommand{\newacr}[4]{\newglossaryentry{glos:#1}{name={#3},description={#4}} \newacronym{acr:#1}{#2}{\gls{glos:#1}} }

% Todo: Einheitlich mit Abkürzungen umgehen.
% Wird die Abkürzung genannt in der Erklärung? Wie oft? Immer? Einmal? Wird der gesamte Name nochmal genannt?

\newacr{nlp}{NLP}{Natural Language Processing}{Natural-language processing (NLP) is the discipline of collecting and analysing natural language. This includes for example speech recognition, natural language understanding and generation \cite{liddy2001natural}.}
\newcommand{\nlp}{\gls{acr:nlp}}

\newacr{uima}{UIMA}{Unstructured Information Management Architecture}{UIMA is a general purpose framework to extract information from unstructured data \cite{uima,OASIS:UIMA:2009}. Although any data format is supported, natural language texts are the most common one.}
\newcommand{\uima}{\gls{acr:uima}}

\newacr{qa}{QA}{Question Answering}{Being a subfield of \nlp{}, Question Answering (QA) is about extracting and understanding questions from natural language and answering them accordingly.}
\newcommand{\qa}{\gls{acr:qa}}

\newacr{dkpro}{DKPro Core}{Darmstadt Knowledge Processing Software Repository}{A collection of UIMA components for natural language processing. This includes analysis engines, language models and custom type systems \cite{dkpro,eckartdecastilho-gurevych:2014:OIAF4HLT}.}
\newcommand{\dkpro}{\gls{acr:dkpro}}

\newacr{uimaas}{UIMA-AS}{UIMA Asynchronous Scaleout}{UIMA-AS is the second generation of UIMA native scaling solutions. It is based on a shared queue based service architecture \cite{uimaas}}
\newcommand{\uimaas}{\gls{acr:uimaas}}

\newacr{cpe}{CPE}{Collection Processing Engine}{Collection Processing Engines (CPE) are the first generation of UIMA native scaling solutions. A CPE contains a collection reader, which knows how to read the underlying collection, and CAS Consumers for the final analysis result extraction \cite{uimacpe}.}
\newcommand{\cpe}{\gls{acr:cpe}}

\newglossaryentry{glos:docker}{name={Docker},description={Docker is a virtualization solution based on containers. By using containers instead of fully fledged virtual machines Docker tries to reduce the system overhead per running application \cite{docker}.}}
\newcommand{\docker}{\gls{glos:docker}}

% TODO: Eigene Texte. Die ab hier sind alle gestohlen :O

\newglossaryentry{glos:spark}{name={Apache Spark},description={Apache Spark is an open-source cluster-computing framework. Originally developed at the University of California, Berkeley's AMPLab, the Spark codebase was later donated to the Apache Software Foundation, which has maintained it since. Spark provides an interface for programming entire clusters with implicit data parallelism and fault tolerance.}}
\newcommand{\spark}{\gls{glos:spark}}
\newglossaryentry{glos:kafka}{name={Apache Kafka},description={Apache Kafka is an open-source stream processing software platform developed by the Apache Software Foundation written in Scala and Java. The project aims to provide a unified, high-throughput, low-latency platform for handling real-time data feeds.}}
\newcommand{\kafka}{\gls{glos:kafka}}
