% !TeX root = Main.tex

% Glossary

% #1: Id (uima)
% #2: Shortname (UIMA)
% #3: Longname (Unstructured Information Management Applications)
% #4: Description (UIMA are software systems that...)
\newcommand{\newacr}[4]{\newglossaryentry{glos:#1}{name={#3},description={#4}} \newacronym{acr:#1}{#2}{\gls{glos:#1}} }

% Todo: Einheitlich mit Abkürzungen umgehen.
% Wird die Abkürzung genannt in der Erklärung? Wie oft? Immer? Einmal? Wird der gesamte Name nochmal genannt?

\newacr{nlpGate}{GATE}{General Architecture for Text Engineering}{}
\newcommand{\nlpGate}{\gls{acr:nlpGate}}
\newacr{xml}{XML}{Extensible Markup Language}{}
\newcommand{\xml}{\gls{acr:xml}}
\newacr{jvm}{JVM}{Java Virtual Machine}{}
\newcommand{\jvm}{\gls{acr:jvm}}
\newacr{api}{API}{Application Programming Interface}{}
\newcommand{\api}{\gls{acr:api}}
\newacr{loc}{LoC}{Lines of Code}{}
\newcommand{\loc}{\gls{acr:loc}}
\newacr{xmi}{XMI}{XML Metadata Interchange}{}
\newcommand{\xmi}{\gls{acr:xmi}}
\newacr{oasis}{OASIS}{Organization for the Advancement of Structured Information Standards}{}
\newcommand{\oasis}{\gls{acr:oasis}}
\newacr{url}{URL}{Uniform Resource Locator}{}
\newcommand{\URL}{\gls{acr:url}}
\newacr{cpm}{CPM}{Collection Processing Manager}{}
\newcommand{\cpm}{\gls{acr:cpm}}
\newacr{stts}{STTS}{Stuttgart-Tübingen-TagSet}{}
\newcommand{\stts}{\gls{acr:stts}}


\newacr{nlp}{NLP}{Natural Language Processing}{Natural-language processing (NLP) is the discipline of collecting and analysing natural language. This includes for example speech recognition, natural language understanding and generation \cite{liddy2001natural}.}
\newcommand{\nlp}{\gls{acr:nlp}}

\newacr{uima}{UIMA}{Unstructured Information Management Architecture}{UIMA is a general purpose framework to extract information from unstructured data \cite{uima,OASIS:UIMA:2009}. Although any data format is supported, natural language texts are the most common one.}
\newcommand{\uima}{\gls{acr:uima}}

\newacr{uimafit}{UIMAfit}{Factories, Injection, and Testing library for UIMA}{}
\newcommand{\uimafit}{\gls{acr:uimafit}}
\newacr{uimacpe}{UIMA-CPE}{UIMA Collection Processing Architecture}{}
\newcommand{\uimacpe}{\gls{acr:uimacpe}}


\newacr{qa}{QA}{Question Answering}{Being a subfield of \nlp{}, Question Answering (QA) is about extracting and understanding questions from natural language and answering them accordingly \cite{jurafsky2014speech}.}
\newcommand{\qa}{\gls{acr:qa}}

\newacr{dkpro}{DKPro Core}{Darmstadt Knowledge Processing Software Repository}{A collection of UIMA components for natural language processing. This includes analysis engines, language models and custom type systems \cite{dkpro,eckartdecastilho-gurevych:2014:OIAF4HLT}.}
\newcommand{\dkpro}{\gls{acr:dkpro}}

\newacr{uimaas}{UIMA-AS}{UIMA Asynchronous Scaleout}{UIMA-AS is the second generation of UIMA native scaling solutions. It is based on a shared queue based service architecture \cite{uimaas}}
\newcommand{\uimaas}{\gls{acr:uimaas}}

\newacr{cpe}{CPE}{Collection Processing Engine}{Collection Processing Engines (CPE) are the first generation of UIMA native scaling solutions. A CPE contains a collection reader, which knows how to read the underlying collection, and CAS Consumers for the final analysis result extraction \cite{uimacpe}.}
\newcommand{\cpe}{\gls{acr:cpe}}

\newglossaryentry{glos:docker}{name={Docker},description={Docker is a virtualization solution based on containers. By using containers instead of fully fledged virtual machines Docker tries to reduce the system overhead per running application \cite{docker}.}}
\newcommand{\docker}{\gls{glos:docker}}


\newacr{ae}{AE}{Analysis Engine}{An Analysis Engine is a component of an \uima{} \nlp{} pipeline. As such it analyses given documents and enriches it with inferred information. An Analysis Engine may contain other Analysis Engines \cite{uimacpe}.}
\newcommand{\anen}{\gls{acr:ae}}
\newcommand{\anens}{\glspl{acr:ae}}

\newacr{cas}{CAS}{Common Analysis System}{The Common Analysis System is a type of object in the \uima{} framework. It contains the subject of analysis, the analysis result and a corresponding type system \cite{uimacpe}.}
\newcommand{\cas}{\gls{acr:cas}}


\newacr{sofa}{SofA}{Subject of Analysis}{The Subject of Analysis is the document that gets analyzed by a given \uima{} application. It is contained in its corresponding \cas{} \cite{uimacpe}.}
\newcommand{\sofa}{\gls{acr:sofa}}



% TODO: Eigene Texte. Die ab hier sind alle gestohlen :O

\newglossaryentry{glos:spark}{name={Apache Spark},description={\marginnote{Gestohlener Text, muss noch paraphrasiert werden.}Apache Spark is an open-source cluster-computing framework. Originally developed at the University of California, Berkeley's AMPLab, the Spark codebase was later donated to the Apache Software Foundation, which has maintained it since. Spark provides an interface for programming entire clusters with implicit data parallelism and fault tolerance.}}
\newcommand{\spark}{\gls{glos:spark}}

\newglossaryentry{glos:kafka}{name={Apache Kafka},description={\marginnote{Gestohlener Text, muss noch paraphrasiert werden.}Apache Kafka is an open-source stream processing software platform developed by the Apache Software Foundation written in Scala and Java. The project aims to provide a unified, high-throughput, low-latency platform for handling real-time data feeds.}}
\newcommand{\kafka}{\gls{glos:kafka}}

\newglossaryentry{glos:hadoop}{name={Apache Hadoop},description={\marginnote{Gestohlener Text, muss noch paraphrasiert werden.}Apache Hadoop is an open-source software framework used for distributed storage and processing of datasets of big data using the MapReduce programming model.}}
\newcommand{\hadoop}{\gls{glos:hadoop}}

\newacr{hdfs}{HDFS}{Hadoop Distributed File System}{\marginnote{Gestohlener Text, muss noch paraphrasiert werden.}The Hadoop Distributed File System (HDFS) is a distributed file system designed to run on commodity hardware.}
\newcommand{\hdfs}{\gls{acr:hdfs}}
