\documentclass[11pt,a4paper,oneside]{report}

% Packages
\usepackage[utf8]{inputenc}		
\usepackage[T1]{fontenc}			
\usepackage[english]{babel}						% Speling Errors? Me? Never!


\usepackage{amsmath}							% For all your math needs...
\usepackage{amsfonts}							% ...and beyond...
\usepackage{amssymb}							% ...and even further.
\usepackage{makeidx}							% We have a number of indexes.
\usepackage{url}								% URLs in bib
\usepackage{graphicx}							% For pictures and stuff.
\usepackage[pdfusetitle]{hyperref}				% Not really necessary.
\usepackage[acronyms,nohypertypes={acronym,main},nopostdot]{glossaries} % For the glossary.
\usepackage[bottom,perpage,multiple]{footmisc}	% Place footnotes at the bottom of the page, restart counting each page and merge multiple footnotes.
\usepackage{csquotes}							% Quoting. Probably only used in the glossary.
\usepackage[strings]{underscore}				% Bibtex is confused by underscores in URLs. Yeah... underscores.
\usepackage{censor}								% For ████ing that ████ed piece of ████. Actually only for development.
\usepackage{listings}							% Code Snippets.
\usepackage{color}								% Colorful code snippets :3
\usepackage[singlelinecheck=off]{caption}


% Dev Only
\usepackage{marginnote}							% For development comments.
\renewcommand*{\marginfont}{\tiny}
\usepackage[marginparsep=3mm, marginparwidth=2cm]{geometry}


% Glossary Styling
\setacronymstyle{short-long}
\setglossarystyle{altlist}
\makenoidxglossaries

% Listings Styling
\definecolor{grey}{rgb}{0.8,0.8,0.8}
\lstset{
	tabsize=2,
	rulecolor=\color{black},
	numberstyle=\color{grey},
	numbersep=10pt,
	framesep=5pt,
	xleftmargin=5pt,
	xrightmargin=5pt,
	numbers=left,
	keepspaces=true,
	frame=shadowbox,
	captionpos=b,
	breaklines=false,
	basicstyle=\footnotesize
}
\lstdefinestyle{YAML}{
	morekeywords={FROM,COPY,ENTRYPOINT,ADD,CMD,HEALTHCHECK}
	}

% Header Imports
\graphicspath{{img/}}
\bibliographystyle{alpha}

\def\A{{\mathcal{A}}}
\def\B{{\mathcal{B}}}
\def\C{{\mathcal{C}}}
\def\D{{\mathcal{D}}}
\def\E{{\mathcal{E}}}
\def\F{{\mathcal{F}}}
\def\G{{\mathcal{G}}}
\def\H{{\mathcal{H}}}
\def\I{{\mathcal{I}}}
\def\K{{\mathcal{K}}}
\def\L{{\mathcal{L}}}
\def\M{{\mathcal{M}}}
\def\N{{\mathcal{N}}}
\def\O{{\mathcal{O}}}
\def\P{{\mathcal{P}}}
\def\Q{{\mathcal{Q}}}
\def\R{{\mathcal{R}}}
\def\S{{\mathcal{S}}}
\def\T{{\mathcal{T}}}
\def\U{{\mathcal{U}}}
\def\V{{\mathcal{V}}}
\def\W{{\mathcal{W}}}
\def\X{{\mathcal{X}}}
\def\Y{{\mathcal{Y}}}
\def\Z{{\mathcal{Z}}}

\def\AA{{\mathbb{A}}}
\def\BB{{\mathbb{B}}}
\def\CC{{\mathbb{C}}}
\def\DD{{\mathbb{D}}}
\def\EE{{\mathbb{E}}}
\def\FF{{\mathbb{F}}}
\def\GG{{\mathbb{G}}}
\def\HH{{\mathbb{H}}}
\def\II{{\mathbb{I}}}
\def\KK{{\mathbb{K}}}
\def\LL{{\mathbb{L}}}
\def\MM{{\mathbb{M}}}
\def\NN{{\mathbb{N}}}
\def\OO{{\mathbb{O}}}
\def\PP{{\mathbb{P}}}
\def\QQ{{\mathbb{Q}}}
\def\RR{{\mathbb{R}}}
\def\SS{{\mathbb{S}}}
\def\TT{{\mathbb{T}}}
\def\UU{{\mathbb{U}}}
\def\VV{{\mathbb{V}}}
\def\WW{{\mathbb{W}}}
\def\XX{{\mathbb{X}}}
\def\YY{{\mathbb{Y}}}
\def\ZZ{{\mathbb{Z}}}
\def\\{{\setminus}}

\newcommand{\footurl}[2]{\footnote{\url{#1}, last accessed on #2.}}
% Glossary

% #1: Id (uima)
% #2: Shortname (UIMA)
% #3: Longname (Unstructured Information Management Applications)
% #4: Description (UIMA are software systems that...)
\newcommand{\newacr}[4]{\newglossaryentry{glos:#1}{name={#3},description={#4}} \newacronym{acr:#1}{#2}{\gls{glos:#1}} }

\newacr{uima}{UIMA}{Unstructured Information Management Applications}{UIMA are software systems that analyze large volumes of unstructured information in order to discover knowledge that is relevant to an end user. An example UIM application might ingest plain text and identify entities, such as persons, places, organizations; or relations, such as works-for or located-at.}

\newcommand{\uima}{\gls{acr:uima}}

\newglossaryentry{glos:dkpro}{%
    name=DKPro,
    description={DKPro is a community of projects focussing on re-usable Natural Language Processing software}
    }

\newcommand{\dkpro}{\gls{glos:dkpro}}


% Meta Data Definitions
\newcommand{\authortext}{Simon Gehring}
\newcommand{\street}{Am Jesuitenhof}
\newcommand{\email}{simon.gehring@fkie.fraunhofer.de}
\newcommand{\housenumber}{3}
\newcommand{\postal}{53117}
\newcommand{\city}{Bonn}
\newcommand{\mn}{2553262}

\newcommand{\doctype}{Master Thesis}
\newcommand{\titletext}{Scaling UIMA}
\newcommand{\subtitletext}{} % Don't know anything interesting.
\newcommand{\field}{Computer Science}
\newcommand{\organization}{Rheinische Friedrich-Wilhelms-Universität Bonn}
\newcommand{\cooperation}{Fraunhofer-Institut für Kommunikation, Informationsverarbeitung und Ergonomie}

\newcommand{\supervisorOne}{Prof.~Dr.~Heiko \textsc{Röglin}}
\newcommand{\supervisorTwo}{\censor{Irgendwer ;_;}}


% Meta Data Compiling
\title{\titletext}
\author{\authortext}
\date{\today}
\newcommand{\address}{\street{} \housenumber{}\linebreak
\postal{} \city{}}

% Document Start
\begin{document}
%\doctype Master Thesis
%\titletext Some Title
%\subtitletext Some Subtitle
%\field Computer Science
%\organization Rheinische Friedrich-Wilhelms-Universität Bonn
%\cooperation Fraunhofer-Institut für Kommunikation, Informationsverarbeitung und Ergonomie
%\address

\begin{titlepage}    
    \centering
	{\scshape\Large \doctype{}\par}
	{\scshape\large \field{}\par}
	\vspace{1.5cm}
	{\Huge\bfseries \titletext{}\par}
	\vspace{1cm}
	{\LARGE \subtitletext{}\par}
	\vspace{2cm}
	{\Large\itshape \authortext{}\par}
	\vspace{0.35cm}
	{\large \address\par}
	{\large \email\par}
	{\large Matriculation Number \mn\par}
	\vfill
	    { At the \par}
	    \vspace{0.1cm}
	    {\scshape\large\organization{}\par}
	    \vspace{0.1cm}

%{in cooperation with the\par}
%    \vspace{0.1cm}
%    {\scshape\large\cooperation{}\par}	
%    \vspace{0.1cm}
%    {and\par}
%    \vspace{0.1cm}
%    {\scshape\large\cooperationTwo{}\par}
	\vfill	
	supervised by\par
	\supervisorOne{} and 
	\supervisorTwo{}
	

	

	\vfill

% Bottom of the page
	{\large \today\par}
\end{titlepage}
\tableofcontents
% Content
\chapter{Introduction}

Natural language is part of everyone's everyday life and is most commonly used to transmit information human-to-human. While most of this interaction takes place orally or written on paper, the digital revolution and the rise of social media increased the amount of digitally stored natural language tremendously. Gantz and Reinsel predicted 2012 that the amount of digital data stored globally will double about every two years until at least the year 2020 \cite{gantz2012digital}.

Many opportunities arise from this amount of digital data, specifically in the field of machine learning. In 2011, IBM's \qa{} system ``Watson'' famously outmatched professional players in the quiz show ``Jeopardy!'' \cite{ferrucci2012introduction,epstein2012making}. Kudesia et al. proposed 2012 an algorithm to detect so called CAUTIs\footnote{Catheter-associated Urinary Tract Infections}, common hospital-acquired infections, by utilizing a \nlp{} analysis on the medical records of patients \cite{kudesia2012natural}.

Apache \uima{} is one of few general approaches to implement \nlp{} solutions. With a very modular architecture, \uima{} is a popular tool that can easily be applied to a majority of \nlp{} problems. A large part of the popularity of \uima{} stems from the large \dkpro{} collection of components, containing hundreds of analysis modules and precomputed language models \cite{eckartdecastilho-gurevych:2014:OIAF4HLT}, which are easily imported into existing Java projects with the build automation tool Apache Maven \cite{dkpro}.

A common problem with \uima{} in non-academic environments is scaling \cite{divita2015scaling,epstein2012making,ramakrishnan2010building}. \uima{} itself provides two distinct interfaces to analyse larger collections of unstructured data, with one being \uimaas{} and the other being the more dated and less flexible \cpe{} \cite{OASIS:UIMA:2009}.

In this thesis, we will evaluate different means of scaling \uima{}, using modern technologies like \docker{}, a container virtualization solution, \spark{}, a cluster computing framework, and \kafka{}, an information stream processing software. We will compare said implementations with the native \uimaas{} and \cpe{} approach in terms of processor and memory efficiency, ease of implementation and maintainability.


\section{Motivation}

\section{Basics}

In this chapter, we will cover the basics for the necessary technologies used throughout the evaluation. All of these are concrete implementations of more general concepts and may be exchanged for similar products. However, the following products were chosen, mainly because they are Open Source \footurl{https://svn.apache.org/viewvc/uima/}{2018-02-27}\footurl{https://github.com/docker}{2018-02-27}\footurl{https://github.com/apache/hadoop}{2018-02-27}\footurl{https://github.com/apache/spark}{2018-02-27}\footurl{https://github.com/apache/kafka}{2018-02-27}
but also because of their popularity and relevance in industry.
\subsection{UIMA}


\subsection{Docker}
\subsection{Hadoop}
\subsection{Spark}
\subsection{Kafka}


\section{Problem}
\subsection{Scaling UIMA}
\subsubsection{UIMA-AS}
\subsubsection{UIMA-CPM}
\subsection{Implementation Requirements}


\section{Related Work}

\subsection{Watson}

\subsection{Something else that warrants another subsection}

\subsection{GATE?}


\section{Outline} 	% Einführung, Motivation, Related Work
\chapter{Implementation}

\section{Concrete Application}

\section{Documents -> Kafka}

\section{Kafka -> Spark}

\section{Spark -> UIMA}

\section{UIMA -> Java}

\section{Kafka -> Output}

\section{Scaling all this}
	% Hauptteil, Lösungen von Problemen, Implementierungsdetail(?)
\chapter{Evaluation}

\section{Computation Speed}

\section{Memory Usage}

\section{Extensibility}

\section{Maintainability}		% Evaluation (Geschwindigkeit, Speicher etc.)
\chapter{Summary}






\section{The Judgement}			% Zusammenfassung / Fazit
\chapter{Future Work}			% Mögliche Future Work

% Glossary obv.
\printnoidxglossary[sort=letter]
% Bibliography obv.
\bibliography{bibliography}

% Eidesstattliche Erklärung des Selbstständigen Verfassens
\pagebreak
\section*{Eidesstattliche Erkl\"arung}
Hiermit versichere ich, Simon Gehring, dass ich die vorliegende Masterarbeit selbstst\"andig verfasst und keine anderen als die angegebenen Quellen und Hilfsmittel benutzt habe. Die Stellen meiner Arbeit, die dem Wortlaut oder dem Sinne nach anderen Werken und Quellen, einschlie\ss{}lich Quellen aus dem Internet, entnommen sind, habe ich in jedem Fall unter Angabe der Quelle deutlich als Entlehnung kenntlich gemacht. Dasselbe gilt sinngem\"a\ss{} f\"ur Tabellen, Karten und Abbildungen.

\vspace*{2cm}
Unterschrift: \hrulefill

\hspace*{0mm}\phantom{Unterschrift: }Simon Gehring, Student

\hspace*{0mm}\phantom{Unterschrift: }Universit\"at Bonn

  
\end{document}
