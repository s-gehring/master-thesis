\chapter{Summary}\label{ch:summary}
The following chapter concludes the thesis by summarizing its results and the frameworks limitations. Furthermore, an outlook on the frameworks source code availability and possible future changes will be given.

\section{Limitations}
Eventuell sogar schon in die Implementierung?

\section{Conclusion}

Das hängt ein wenig von den Ergebnissen ab. Ich hoffe auf sowas wie "Ich bin der tollste, UIMA-AS ist Dreck". Das spiegeln die Daten zwar nicht ganz wieder, aber hey :D


\section{Availability}
\label{sec:availability}
From November 2018 on, the frameworks code will be publicized on GitHub\footnote{On \url{https://github.com/s-gehring/master-thesis-program}}. Since the framework is wrapped inside a maven project, it will also be uploaded to the central maven repository. Furthermore, another git repository that contains a working \spark{} dockerfile and the evaluation setup architecture will be published\footnote{On \url{https://github.com/s-gehring/master-thesis-spark}}. Next, the maven project that defines the benchmarking Java code will be available\footnote{On \url{https://github.com/s-gehring/master-thesis-benchmark}}. A hybrid project that consists of the deployment of \uimaas{} used in the evaluation and defines a dockerfile containing a working \uimaas{} installation will also be published on GitHub\footnote{On \url{https://github.com/s-gehring/master-thesis-uimaas}}. At last, this Thesis and the corresponding \LaTeX{} code will be available on GitHub as well \footnote{On \url{https://github.com/s-gehring/master-thesis}}.

All repositories will be published under the MIT license and are therefore free to use.

\section{Outlook}
First, the framework and all evaluation related code and resources will be made public according to Section~\ref{sec:availability}. Other than that further improvements to the presented framework can still be made. The wrapping class \lstinline|AnalysisResult| only provides a subset of its underlying \lstinline|JavaRDD| functionality. The other functions were not needed at the time of writing and have therefore been neglected. However, an unwrapping of said \rdd{} may be desired, making further processing of the underlying \cas{} possible. In such a way, one could benefit substantially more from \spark{}s optimization features.

Furthermore, more compression algorithms may be implemented in the future, making compression for different serializations feasible. New serialization techniques, like delta \cas{} as used in \cite{epstein2012making} could also help improve the frameworks performance.