% !TeX root = ../Main.tex

\chapter{Introduction}


Natural language is most commonly used to transmit information human-to-human. While most of this interaction takes place orally or written on paper, the digital revolution and the rise of social media increased the amount of digitally stored natural language tremendously. Gantz and Reinsel predicted 2012 that the amount of digital data stored globally will double about every two years until at least the year 2020 \cite{gantz2012digital}.

Many opportunities arise from this amount of digital data, specifically in the field of machine learning. In 2011, IBM's \qa{} system ``Watson'' famously outmatched professional players in the quiz show ``Jeopardy!'' \cite{ferrucci2012introduction,epstein2012making} and is now used in the \emph{Crew Interactive Mobile Companion} (CIMON), an AI aiding with various tasks in space \cite{cimon}. Kudesia et al. proposed 2012 an algorithm to detect so called CAUTIs\footnote{Catheter-associated Urinary Tract Infections}, common hospital-acquired infections, by utilizing a \nlp{} analysis with precomputed language models on the medical records of patients \cite{kudesia2012natural}.

Current projections suggest a growth of revenues from the natural language processing market in North America, Western Europe, and the Asia-Pacific region of at least a factor of three until 2024 \cite{stat:nlpeurope,stat:nlpamerica,stat:nlpasia}.

\section{Motivation}

Natural language is inherently unstructured and hardly machine readable. Even a seemingly easy task like separating a sentence into words is still an ongoing research topic \cite{pak2018text}. Since many \nlp{} frameworks like Stanford's Core \nlp Suite \cite{manning-EtAl:2014:P14-5}, The Natural Language Toolkit \cite{bird2004nltk} and Apache OpenNLP \cite{opennlp} exist, there was a need for generalizing the \nlp{} approach. In 1995, the University of Sheffield began developing \nlpGate{}, a graphical development application for generic \nlp{} problems \cite{cunningham2002gate}. Aside from algorithms for typical \nlp{} tasks like tokenization, sentence splitting or part of speech tagging, \nlpGate{} provides a graphical interface for developing custom algorithms in form of plugins, which can use the results of previous \nlp{} analyses. However, the possibilities for scaling \nlpGate{} applications are sparse. In 2012 GATECloud.net launched, a proprietary, cloud based \nlpGate{} computation interface \cite{tablan2013gatecloud}. Since then, no improvements to the scaling capability of \nlpGate{} has been made. 

In \cite{ferrucci2004uima}, Ferrucci and Lally introduced \uima{}, another general purpose \nlp{} framework. Initially developed by IBM, \uima{} has been open-source and is maintained by Apache since 2006. \uima{} itself provides no built-in \nlp{} analyses, but offers a common analysis structure among its plugins, which is used to combine different \nlp{} approaches into one single framework. A popular implementation example of \uima{} is IBM's \qa{} system ``Watson'' \cite{ferrucci2012introduction,epstein2012making}, as described above. For this matter, \uima{} was configured to run on thousands of processor cores to achieve a feasible reaction time \cite{epstein2012making}. Although this example seems to demonstrate that Apache \uima{} is indeed very scalable, it has its drawbacks. The native \uima{} scaling framework \uimaas{} is configured by \xml{} files, which are difficult to maintain. Furthermore, it does not work out-of-the-box with other \uima{} derivatives like \uimafit{}. 

In this thesis we will implement and evaluate a scale-out framework for \uima{}, which utilizes modern technology to ensure maintainability and scalability.

\section{Implementation Requirements}
\label{sec:requirements}
The implementation should meet a number of requirements. First and foremost, the underlying framework, Apache \uima{}, must not be limited in functionality. Since one of \uima{}s strengths is the modularity and ease of plugin development, the scale-out framework to implement is required to work with native \uima{} classes without any restriction. While this requirement sounds easy to meet, it is actually quite limiting since \uima{} plugins can be arbitrary Java code. Without special care, such code is not necessarily thread-safe and thus can not be safely executed multiple times in parallel within one \jvm{}.

The metrics, the scale-out framework should optimize, include:
\begin{itemize}
\item CPU Usage
\item RAM Usage
\item Document Throughput
\item Byte Throughput
\item Maintainability
\item Implementability
\item Code Quality
\end{itemize}
With CPU and RAM Usage being the proportion of actually used (virtual or real) resources, higher values are more preferable. Even if many of those resources go into administrative tasks, a scaling framework should use as many of the available resources as possible, when faced with a large list of parallel jobs. Thus, a value near one should be aimed for. At the same time, resource allocation when idling should be as low as possible.

A little more obvious metric is the achieved throughput of data. Since collections of input documents can be shaped very differently, both, a high document and byte throughput are desired. When handling small documents, maybe even single words or sentences, the byte throughput is limited by the actual input size and the large number of documents is responsible for the size of the input data collection. In such a situation, a high document throughput would be preferred to a high byte throughput. On the other hand, on larger documents, a high byte throughput is the more accurate metric, since it is independent of the individual size of the input documents.

The framework is aimed to work in large academic but also industry compliant environments. Therefore, the maintainability is important. It describes the amount of effort to maintain any current usage of the framework, for example changing the underlying \nlp{} algorithm, modifying the hardware setup or making configuration changes. This is inherently more complicated by the genericness of \uima{}, which allows for sophisticated plug-in initialization and configuration logic, making on-the-fly changes more difficult.

Furthermore, the framework should be easy to utilize. Code that already has been written for single-threaded execution should be easily reusable. This is especially important for \uima{} since large repositories of plug-ins like the \dkpro{} \cite{dkpro} already exist and are infeasible to be rewritten. 

Equally important as ease of utilization is the longevity of the framework. Ensuring a maximum of Code Quality is necessary to avoid large refactorings and \api{} changes in the near future. In a non-academic environment, applications often have to last for a long time before replacement. Thus, a robust underlying code is desirable.

The last three metrics maintainability, implementability and code quality, are not trivial to measure, since those are subject to individual perception. However, implementability can be measured in \loc{} needed to achieve a common use case like deploying a predefined \nlp{} algorithm. There are further static code analyzers, which return an amount of code quality issues. Such a score can be used to measure the quality of the framework's code. This leaves maintainability as the only attribute that can not easily be measured in numbers. However, large differences between the different approaches become clear in Chapter~\ref{ch:evaluation}, making a comparison possible.

\section{Outline}
First, in Chapter~\ref{ch:basics}, the functionality of \uima{} is detailed, especially with focus on distributed computing and scaling. For this matter, both native \uima{} scaling frameworks, \uimaas{} and \cpe{} are subject in said chapter. Also \spark{}, a cluster-computing framework, will be briefly explained since it will form the foundation of the frameworks scaling capabilities. \docker{} follows, which is heavily used in the evaluation as virtualization technique. Then, approaches like Leo and v3NLP will be introduced. Those are also \uimaas{}-based scaling solutions which try to solve the problems commonly occurring with native \uimaas{}.

Chapter~\ref{ch:implementation} will explain the scaling framework in detail. It starts with \spark{} as the chosen technology and then proceeds to the implementation. A special focus lies on the data distribution between different worker machines.

Then, Chapter~\ref{ch:evaluation} deals with the results of the framework implementation. For this, the requirements given in Section~\ref{sec:requirements} are evaluated against the framework, \uimaas{} and the single-threaded approach. A sophisticated setup, which is also depicted in the chapter, was necessary to provide comparable result.

Lastly, Chapter~\ref{ch:summary} summarizes the results, including possible limitations of the implementation. It also gives an outlook on how to extend and publicize the framework.