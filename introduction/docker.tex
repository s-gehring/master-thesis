% !TeX root = ../Main.tex

\subsection{Docker}
\marginnote{Source?}\docker{} is a software to create and run applications in virtual environments, without the need to setup a complete virtual operating system for each application. Based on Cgroups and namespaces, \docker{} runs applications in a containerized environment.
To create a containerized application, a special markup file, called a Dockerfile has to be written by the programmer. The Dockerfile defines exactly how the applications' environment needs to look like. Listing \ref{dockerfile} shows how such a Dockerfile typically looks like. Based on an Ubuntu image, it installs Java via apt-get. Afterwards it copies the file {\em{} application.jar} into the root of the image and tells Docker to execute it via \lstinline[]|java -jar /app.jar -Xmx4G -Xms4G|. Having such a Dockerfile ensures reproducibility of the constructed images. After building said Dockerfile, an image is created.\marginnote{source für Registries? Öffentliches Docker-Registry?} This image can be serialized into a file but is usually uploaded to Docker repositories, called registries. 

\begin{minipage}{\textwidth}
	
\begin{lstlisting}[style=YAML,caption=A sample Dockerfile used for creating a simple Ubuntu based image to start a Java application.,label=dockerfile]
FROM	ubuntu

RUN		apt(*@-@*)get update
RUN		apt(*@-@*)get install openjdk(*@-@*)9(*@-@*)jre

COPY	./application.jar /application.jar

ENTRYPOINT ["java", "(*@-@*)jar /app.jar", "(*@-@*)Xmx4G", "(*@-@*)Xms4G"]
\end{lstlisting}

\end{minipage}

\marginnote{SOUUUURCE!!}If published through a registry, Docker automatically downloads images on demand. Since the application as well as the whole environment necessary to run it lie within the image, it can be executed. Docker creates a container, a virtual environment based on the given image, \marginnote{Zweimal within? Thesaurus lässt grüßen.}and runs the application within. Containers are usually not aware of each other, thus the same application can be started multiple times, just by starting more containers. \marginnote{An den Haaren herbeigezogen.}This modular view on containers will play a fundamental role on scaling.