\chapter{Introduction}

Natural language is part of everyone's everyday life and is most commonly used to transmit information human-to-human. While most of this interaction takes place orally or written on paper, the digital revolution and the rise of social media increased the amount of digitally stored natural language tremendously. Gantz and Reinsel predicted 2012 that the amount of digital data stored globally will double about every two years until at least the year 2020 \cite{gantz2012digital}.

Many opportunities arise from this amount of digital data, specifically in the field of machine learning. In 2011, IBM's \qa{} system ``Watson'' famously outmatched professional players in the quiz show ``Jeopardy!'' \cite{ferrucci2012introduction,epstein2012making}. Kudesia et al. proposed 2012 an algorithm to detect so called CAUTIs\footnote{Catheter-associated Urinary Tract Infections}, common hospital-acquired infections, by utilizing a \nlp{} analysis on the medical records of patients \cite{kudesia2012natural}.

Apache \uima{} is one of few general approaches to implement \nlp{} solutions. With a very modular architecture, \uima{} is a popular tool that can easily be applied to a majority of \nlp{} problems. A large part of the popularity of \uima{} stems from the large \dkpro{} collection of components, containing hundreds of analysis modules and precomputed language models \cite{eckartdecastilho-gurevych:2014:OIAF4HLT}, which are easily imported into existing Java projects with the build automation tool Apache Maven \cite{dkpro}.

A common problem with \uima{} in non-academic environments is scaling \cite{divita2015scaling,epstein2012making,ramakrishnan2010building}. \uima{} itself provides two distinct interfaces to analyze larger collections of unstructured data, with one being \uimaas{} and the other being the more dated and less flexible \cpe{} \cite{OASIS:UIMA:2009}.

In this thesis, we will evaluate different means of scaling \uima{}, using modern technologies like \docker{}, a container virtualization solution, \spark{}, a cluster computing framework, and \kafka{}, an information stream processing software. We will compare said implementations with the native \uimaas{} and \cpe{} approach in terms of processor and memory efficiency, ease of implementation and maintainability. The evaluation will be based on a specific scenario, however it will be easily configurable by exchanging very few lines of code. This scenario and the corresponding data set has yet still to be found.


\section{Motivation}
\marginnote{Das wird eine kleine section. Das meiste wird aus Introduction rausgenommen. Und der Text direkt unter Introduction ist als Einleitung in das Kapitel zu verstehen und daher nur 1-3 Absätze groß. Einleitungception.}


\section{Problem}
\subsection{Scaling UIMA}
\subsubsection{UIMA-CPM}
\subsubsection{UIMA-AS}
\subsubsection{UIMA-DUCC}
\subsection{Implementation Requirements}


\section{Related Work}

\subsection{Watson}

\subsection{v3NLP}

\subsection{GATE?}

\section{Outline}