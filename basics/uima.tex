% !TeX root = ../Main.tex

\section{UIMA}
\uima{} is a data mining and \nlp{} framework, created in 2005 by IBM \cite{ferrucci2004uima} and maintained by Apache since 2006 \cite{uimacpe}. It is available in Java and C++ and contains various scale-out options. 
Strictly speaking, one has to differentiate between the \uima{} specification and Apache \uima{}, an open source implementation of said specification \cite{OASIS:UIMA:2009}. Since both terms are often used interchangeably, this thesis will also not differentiate between Apache \uima{} and \uima{}. Unless specified else, we will always reference the Apache \uima{} implementation.

\marginnote{Alles ein wenig wirr. Braucht rewrite. Weiß aber nicht genau wie tief in die Materie ich hier gehen muss.}
The \uima{} specification describes \nlp{} applications as a collection of (mostly) independent components. Such a component is called an \anen{} and enriches a given document by inferred information. To modularize said components, \uima{} provides the notion of a \cas{}. A \cas{} is an object containing the \sofa{}, the analysis results and the used type system.


