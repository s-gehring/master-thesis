% !TeX root = ../Main.tex

\chapter{Basics}

Es gibt haufenweise Konzepte, die ich für das Framework, allerdings noch eher für das Testen und Benchmarken verwendet habe. Diese muss ich natürlich alle (oder zumindest die meisten) vorstellen.

Ganz oben dabei ist natürlich UIMA, wobei das eigentlich schon extrem in der Introduction benutzt wird. Ich mag es nicht jetzt erst auf UIMA einzugehen. Aber was will man machen? Irgendeine Struktur braucht man ja.

Danach kommt Spark. Da ist auf dieser Seite eher \"Benutzer\" als Developer bin, werde ich vermutlich nicht besonders tief in die Materie eingehen. Ich denke ich werde nicht bis zum Map-Reduce kommen. Und wenn dann werde ich das nur anschneiden und auf Quellen verweisen. Die internen Spark-Konzepte sind nunmal nicht wirklich wichtig für die Implementierung.

Selbstverständlich kann ich noch weiter ausholen und erst ein wenig über Java, Maven etc. reden, aber ich glaube das ist etwas overkill. Wer meine Arbeit liest, sollte zumindest Java kennen.

Kommen wir zu dem Versuchsaufbau, kommt natürlich Docker ins Spiel. Docker spielt natürlich im Development (schnelles Deployment, etc.) eine Rolle, hat aber bei mir außerdem die Rolle des VM-Ersatzes übernommen. Wie wichtig Docker für den gesamten Versuch war, lässt sich leicht in den Docker- und Composefiles lesen. Diese garantieren darüber hinaus natürlich Vergleichbarkeit.

Trotzdem sind die underlying Konzepte des Docker-Universums nicht allzu wichtig für das Framework an sich. Deployed wird es (da es sich ja nur um eine library handelt und nicht standalone arbeitet) sowieso nur über Maven. Ich plane daher auch eine mögliche Docker-Section nicht allzu groß zu gestalten.

Ich habe es zwar nicht benutzt, aber eine kleine Einführung in HDFS könnte nützlich sein. Ich werde häufiger (im Kontext von BigData) anmerken, dass Daten vermutlich von einem verteiltem Dateisystem, etwa einem HDFS kommen und dahin geschrieben werden.

%In this chapter, we will cover the basics for the necessary technologies used throughout the evaluation. All of these are concrete implementations of more general concepts and may be exchanged for similar products. However, the following products were chosen, mainly because they are Open Source\footurl{https://svn.apache.org/viewvc/uima/}{2018-02-27}\footurl{https://github.com/docker}{2018-02-27}\footurl{https://github.com/apache/hadoop}{2018-02-27}\footurl{https://github.com/apache/spark}{2018-02-27}\footurl{https://github.com/apache/kafka}{2018-02-27} but also because of their popularity and relevance in the industry.

% % !TeX root = ../Main.tex

\subsection{UIMA}
\uima{} is a data mining and \nlp{} framework, created in 2005 by IBM \cite{ferrucci2004uima} and maintained by Apache since 2006 \cite{uimacpe}. It is available in Java and C++ and contains various scale-out options. 
Strictly speaking, one has to differentiate between the \uima{} specification and Apache \uima{}, an open source implementation of said specification \cite{OASIS:UIMA:2009}. Since both terms are often used interchangeably, this thesis will also not differentiate between Apache \uima{} and \uima{}. Unless specified else, we will always reference the Apache \uima{} implementation.

\marginnote{Alles ein wenig wirr. Braucht rewrite. Weiß aber nicht genau wie tief in die Materie ich hier gehen muss.}
The \uima{} specification describes \nlp{} applications as a collection of (mostly) independent components. Such a component is called an \anen{} and enriches a given document by inferred information. To modularize said components, \uima{} provides the notion of a \cas{}. A \cas{} is an object containing the \sofa{}, the analysis results and the used type system.



% % !TeX root = ../Main.tex

\subsection{\docker{}}
\marginnote{Source?}\docker{} is a software to create and run applications in virtual environments, without the need to setup a complete virtual operating system for each application. Based on Cgroups and namespaces, \docker{} runs applications in a containerized environment.
To create a containerized application, a special markup file, called a Dockerfile has to be written by the programmer. The Dockerfile defines exactly how the applications' environment needs to look like. Listing \ref{dockerfile} shows how such a Dockerfile typically looks like. Based on an Ubuntu image, it installs Java via apt-get. Afterwards it copies the file {\em{} application.jar} into the root of the image and tells \docker{} to execute it via \lstinline[]|java -jar /app.jar -Xmx4G -Xms4G|. Having such a Dockerfile ensures reproducibility of the constructed images. After building said Dockerfile, an image is created.\marginnote{source für Registries? Öffentliches \docker{}-Registry?} This image can be serialized into a file but is usually uploaded to \docker{} repositories, called registries. 

\begin{minipage}{\textwidth}
	
\begin{lstlisting}[style=YAML,caption=A sample Dockerfile used for creating a simple Ubuntu based image to start a Java application.,label=dockerfile]
FROM	ubuntu

RUN		apt(*@-@*)get update
RUN		apt(*@-@*)get install openjdk(*@-@*)9(*@-@*)jre

COPY	./application.jar /application.jar

ENTRYPOINT ["java", "(*@-@*)jar /app.jar", "(*@-@*)Xmx4G", "(*@-@*)Xms4G"]
\end{lstlisting}

\end{minipage}

\marginnote{SOUUUURCE!!}If published through a registry, \docker{} automatically downloads images on demand. Since the application as well as the whole environment necessary to run it lie within the image, it can be executed. \docker{} creates a container, a virtual environment based on the given image, \marginnote{Zweimal within? Thesaurus lässt grüßen.}and runs the application within. Containers are usually not aware of each other, thus the same application can be started multiple times, just by starting more containers. \marginnote{An den Haaren herbeigezogen.}This modular view on containers will play a fundamental role on scaling.
% % !TeX root = ../Main.tex

\section{Hadoop}
\marginnote{Sauce!}\hadoop{} is an open-source framework for computation and storage, distributed over an \marginnote{almost?}(almost) arbitrary number of machines. It contains the \hdfs{}, a distributed file system built for high throughput and reliability, and specifically designed to run on commodity hardware. These properties allow for large clusters of relatively cheap hardware.

\hadoop{} also provides an API for distributed computing, named Hadoop MapReduce, based on the MapReduce algorithm by Dean and Ghemawat in \cite{dean2008mapreduce}. In this algorithm the underlying transformation of data is abstracted into two general steps, Map and Reduce. With $K,L,V,W$ being sets the Map function implements 
\[\mathtt{Map}: K\times{}V\to{}(L\times{}W)^*\]
Thus it maps a key-value pair to a list of key-value pairs of arbitrary length \cite{wiki:mapreduce}. The resulting list of key-value pairs acts as intermediate values. The Reduce function aggregates those intermediate values with
\[\mathtt{Reduce}: L\times{}W^*\to{}\] huh...
The programmer in question needs to understand this concept and divide their code into those two steps.
% \section{Spark}
% \section{Kafka}
